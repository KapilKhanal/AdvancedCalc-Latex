\documentclass{article}
\usepackage[utf8]{inputenc}
\usepackage[english]{babel}

\usepackage{minted}
\usepackage{xcolor}

\definecolor{LightGray}{gray}{0.9}
%\definecolor{DarkGray}{gray}{0.1}

%\pagecolor{DarkGray}

\usemintedstyle{borland}

%New colors defined below
\definecolor{codegreen}{rgb}{0,0.6,0}
\definecolor{codegray}{rgb}{0.5,0.5,0.5}
\definecolor{codepurple}{rgb}{0.58,0,0.82}
\definecolor{backcolour}{rgb}{0.95,0.95,0.92}

\title{Advanced Calculus I}
\author{Assignment-I : Kapil Khanal }
\date{ }

\begin{document}

\maketitle

\section{QUESTION 4b}


\begin{given}
Suppose $y \in f (C \cup D)$ 
    then since for a function $f(x) = y$ ,
    there exists $x \in C \cup D$ 
\end{given}

\begin{implications}
Suppose $x \in C$  or $x \in D$
    if $x \in C$ then $y= f(x) \in f(C)$ \newline
Similarly  if $x \in D$ then $y= f(x) \in f(D)$ \newline
Therefore, $y \in f(C) or y \in f(D)$ \\
    $y \in f(C) \cup f(D)$

\end{implications}

\begin{b}
$f(x) = \{f(x) \mid x \in X \} $ \newline
$f(C) = \{f(x) \mid x \in C\} $ \newline
$f(D) = \{f(x) \mid x \in D\} $ \newline
$f(C\cap D) = \{f(x) \mid x \in C\hat D \} $

Let, C = $\{1,2,3\}$ , D= $\{4,5,6\}$ \newline
$C\capD  = \{1,2,3,4,5,6\}$ \newline
$f(C\capD) = \{1,4,9,16,25,36\}$ \newline
$f(C)\cap f(D) = \phi$
$f(C\capD) = \{1,4,9,16,25,36\}$
\end{b} 

\section{QUESTION 6b}
\begin{c}

$A\cup(B\capC) = (A\cup B)\cap(A\cup C)$
From the left hand side, \newline
$x \in A or (x \in B and x \in C)$
Using the distributive law, \newline 
$(x \in A or x\in B) and (x \in A or x \in C)$ \newline
$x \in (A \cup B) \cap (A \cup C)$
Hence, RHS = LHS proved. 

\end{c}

\section{QUESTION 13}
\begin{d}
Assume, \newline 
p(n) = $\frac{1}{1\cdot2}$  + $\frac{1}{2\cdot3} ..... $ + $\frac{1}{n\cdot(n+1)} $ = $\frac{n}{n+1}$

from mathematical induction, \newline 
Base case is $n = 1 $,\newline 
$\frac{1}{1\cdot2}$ = $\frac{1}{1+1}$ \newline 
hence, Base case verified, 
Now, let's assume the statement $p(n)$ holds true, then by mathematical induction $p(n+1)$ should hold true as well. 
p(n+1) = $\frac{1}{1\cdot2}$  + $\frac{1}{2\cdot3}...  $ + $\frac{1}{n\cdot(n+1)}$ + $\frac{1}{n\cdot(n+1)}$ = $\frac{n}{n+1}$ for all n \in N

 = $\frac{n}{n+1}$ + $\frac{1}{n\cdot(n+1)}$
 = $\frac{n(n+2)+1}{(n+1)(n+2)}$
 = $\frac{n^2 + 2n+1}{(n+1)(n+2)}$ \newline 
 = $\frac{n+1}{n+1+1}$\newline 
 Hence, from mathematical induction this holds true
 
\end{d}

\end{document}
